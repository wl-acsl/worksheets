\documentclass[12pt,letterpaper,fleqn]{article}

\usepackage{amsmath}

\begin{document}

\begin{center}
  Prefix/Infix/Postfix Notation Worksheet

  Created by Sam Craig for the West Lafayette High School ACSL Club of 2014--2015
\end{center}

In prefix, the operator comes before the operands; infix between; and postfix after.
Only infix has parentheses, as the operator will always only have two operands in the other two notations.
$\uparrow$ designates an exponent such that $e^x$ in prefix is $\uparrow e\ x$.
Convert square and n-th roots $\sqrt[n]{x}$ to, in prefix, $\uparrow x\ /\ 1\ n$.

\bigskip
\noindent \textbf{Questions}

\begin{enumerate}

\item Translate the following equation into prefix.
  \[
  E = mc^2
  \]

\end{enumerate}

\pagebreak
\noindent \textbf{Answers}

\begin{enumerate}

\item \begin{align*}
  E = mc^2 &\rightarrow (E = (m \times (c \uparrow 2))) \\
  &\rightarrow (= E\ (\times\ m\ (\uparrow c\ 2))) \\
  &\rightarrow\ = E \times m \uparrow c\ 2
\end{align*}

\end{enumerate}

\end{document}
