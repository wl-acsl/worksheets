\documentclass[12pt,letterpaper,fleqn]{article}

\usepackage{amsmath}

\begin{document}

\begin{center}
  Prefix/Infix/Postfix Notation Worksheet

  Created by Sam Craig for the West Lafayette High School ACSL Club of 2014--2015
\end{center}

In prefix, the operator comes before the operands; infix between; and postfix after.
Only infix has parentheses, as the operator will always only have two operands in the other two notations.
$\uparrow$ designates an exponent such that $e^x$ in prefix is $\uparrow e\ x$.
Use asterisks (*) instead of times signs ($\times$) when using prefix or postfix.
Convert square and n-th roots $\sqrt[n]{x}$ to, in prefix, $\uparrow x\ /\ 1\ n$.
ACSL will probably not have a problem on their test including a square root.

\bigskip
\noindent \textbf{Questions}

\begin{enumerate}

\item Translate the following equation into prefix.
  \[
  E = mc^2
  \]

\item Translate the following equation into postfix.
  \[
  x = \frac{\sqrt{b^2 - 4ac} - b}{2a}
  \]

\item Translate the following prefix expression into infix.
  \[
  +\ A - B \uparrow C\ D
  \]

\item Evaluate the following postfix expression with $A = 10, B = 5, C = 15$.
  \[
  A\ A\ B + C\ A\ B\ / - +\ *
  \]

\end{enumerate}

\pagebreak
\noindent \textbf{Answers}

\begin{enumerate}

\item \begin{align*}
  E = mc^2 &\rightarrow (E = (m \times (c^2))) \\
  &\rightarrow (E = (m \times (c \uparrow 2))) \\
  &\rightarrow (= E\ (\times\ m\ (\uparrow c\ 2))) \\
  &\rightarrow\ = E \times m \uparrow c\ 2 \\
  &\rightarrow\ = E * m \uparrow c\ 2
\end{align*}

\item \begin{align*}
  x = \frac{\sqrt{b^2 - 4ac} - b}{2a} &\rightarrow \left(x = \left(\frac{\left(\left(\sqrt{((b^2) - (4ac))}\right) - b\right)}{(2a)}\right)\right) \\
  &\rightarrow \left(x = \left(\frac{\left(\left(((b \uparrow 2) - (4ac)) \uparrow (1/2) \right) - b\right)}{(2a)}\right)\right) \\
  &\rightarrow (x = (((((b \uparrow 2) - (4ac)) \uparrow (1/2)) - b) / (2a))) \\
  &\rightarrow (x = (((((b \uparrow 2) - (4 \times a \times c)) \uparrow (1/2)) - b) / (2 \times a))) \\
  &\rightarrow (x\ (((((b\ 2 \uparrow)\ (4\ a \times c\ \times)\ -)\ (1\ 2\ /) \uparrow)\ b\ -)\ (2\ a\ \times)\ /) =) \\
  &\rightarrow x\ b\ 2 \uparrow 4\ a \times c\ \times -\ 1\ 2\ / \uparrow\ b\ - 2\ a\ \times\ / = \\
  &\rightarrow x\ b\ 2 \uparrow 4\ a * c\ * -\ 1\ 2\ / \uparrow\ b\ - 2\ a\ *\ /
\end{align*}

\item \begin{align*}
  +\ A - B \uparrow C\ D &\rightarrow (+\ A\ (-\ B\ (\uparrow C\ D))) \\
  &\rightarrow (A + (B - (C \uparrow D))) \\
  &\rightarrow A + B - C^D
\end{align*}

\item \begin{align*}
  A\ A\ B + C\ A\ B\ / - +\ * &\rightarrow (A\ ((A\ B\ +)\ (C\ (A\ B\ /)\ -)\ +)\ *) \\
  &\rightarrow (10\ ((10\ 5\ +)\ (15\ (10\ 5\ /)\ -)\ +)\ *) \\
  &\rightarrow (10\ (15\ (15\ 2\ -)\ +)\ *) \\
  &\rightarrow (10\ (15\ 13\ +)\ *) \\
  &\rightarrow (10\ 28\ *) \\
  &\rightarrow 280
\end{align*}

\end{enumerate}

\end{document}
