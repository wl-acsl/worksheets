\documentclass[12pt,letterpaper,fleqn]{article}

\usepackage{amsmath}

\begin{document}

\begin{center}
  Recursive Functions Worksheet

  Created by Sam Craig for the West Lafayette High School ACSL Club of 2014--2015
\end{center}

Recursive functions below are defined as they would be in mathematics.
For example, $f(x) = x + 5$ is an example of a function definition.
An example of a definition of a recursive function is
\begin{align*}
  f(x) =
  \begin{cases}
    f(f(x-2)) + 1 & \text{when}\ x > 1 \\
    2 & \text{when}\ x = 1 \\
    1 & \text{when}\ x = 0
  \end{cases}
\end{align*}

\noindent \textbf{Questions}

\begin{enumerate}

\item Consider the following recursive algorithm for painting a rectangle:
  
  \begin{enumerate}

  \item Given a rectangle.

  \item If the width of the rectangle is less than 1 foot, then stop.

  \item Divide the rectangle into 4 equal area rectangles by keeping the height consistent and splitting at the width.
    For example, if a rectangle of dimensions 10x3 was being painted, the four rectangles created would have dimensions 2.5x3.

  \item Paint two of these new rectangles.

  \item Repeat this procedure (start at step 1) for the two unpainted rectangles.

  \end{enumerate}

  How many square feet will be painted when this algorithm is applied to a rectangle of dimensions 24x12 feet?

\item Evaluate $f(8)$, given

  \begin{align*}
    f(x) =
    \begin{cases}
      f(x + 5) & \text{when}\ x < 20 \\
      2(f(x - 1)) + 3 & \text{when}\ x > 20 \\
      x + 2 & \text{when}\ x = 20
    \end{cases}
  \end{align*}

\item Evaluate $f(22)$, given

  \begin{align*}
    f(x) =
    \begin{cases}
      f(2x - 5) / 2 & \text{when}\ x < 100 \\
      x^2 - x & \text{otherwise}
    \end{cases}
  \end{align*}

\item Evaluate $f(14, 7)$, given

  \begin{align*}
    f(x, y) =
    \begin{cases}
      f(x - 3, y - 1) & \text{when}\ x > y \\
      x + y & \text{otherwise}
    \end{cases}
  \end{align*}

\item Evaluate $f(20, 13)$ given

  \begin{align*}
    f(x, y) =
    \begin{cases}
      f(x - 3, 2y - x + 5) + 3 & \text{when}\ x > y \\
      2x - y & \text{otherwise}
    \end{cases}
  \end{align*}

\end{enumerate}

\pagebreak

\noindent \textbf{Answers}

\begin{enumerate}

  %% Consider using a shape.
  %% It would be a fun way to show this!
\item Four rectangles of dimensions 6x12 feet are created.
  Two are painted, and two remain.
  These two yield rectangles of dimensions 1.5x12 feet.
  Since eight rectangles of this dimension are created, four will be painted.
  The next rectangles created will have width of 0.375 feet, so none of them will be painted.

  Therefore, the total square feet is $2(6\times 12) + 4(1.5\times 12) = 216$ square feet.

\item \begin{align*}
  f(8) &= f(13) \\
  &= f(18) \\
  &= f(23) \\
  &= 2(f(22)) + 3 \\
  &= 2(2(f(21)) + 3) + 3 \\
  &= 2(2(2(f(20)) + 3) + 3) + 3 \\
  &= 2(2(2(22) + 3) + 3) + 3 \\
  &= 2(2(47) + 3) + 3 \\
  &= 2(97) + 3 \\
  &= 197
\end{align*}

\item \begin{align*}
  f(22) &= f(39) / 2 \\
  &= (f(73) / 2) / 2 \\
  &= ((f(141) / 2) / 2) / 2 \\
  &= ((19740 / 2) / 2) / 2 \\
  &= 2467.5
\end{align*}

\item \begin{align*}
  f(14, 7) &= f(11, 6) \\
  &= f(8, 5) \\
  &= f(5, 4) \\
  &= f(2, 3) \\
  &= 5
\end{align*}

\item \begin{align*}
  f(20, 13) &= f(17, 11) + 3 \\
  &= (f(14, 10) + 3) + 3 \\
  &= ((f(11, 11) + 3) + 3) + 3 \\
  &= ((11 + 3) + 3) + 3 \\
  &= 20
\end{align*}

\end{enumerate}

\end{document}
