\documentclass[12pt,letterpaper]{article}

\usepackage{amsmath}

\begin{document}

\begin{center}
  LISP Programming Worksheet

  Created by Christopher Cooper and Sam Craig for the West Lafayette High School ACSL Club of 2014--2015
\end{center}

\textbf{You may not need to know all of this material for the next contest!}
Problems 1 and 2 are representative of problems you may see on contest \#2.
Problems 3 through 5 are representative of problems you may see on contest \#4.

\bigskip
\noindent \textbf{Questions}

Evaluate the following LISP code.

\begin{enumerate}

\item
\begin{verbatim}
(DIV 24 (SUB 2 5))
\end{verbatim}

\item
\begin{verbatim}
(ADD (MULT 48 40)
     (SUB (SQUARE 9) 1)
     (MULT 3 5))
\end{verbatim}

\item
\begin{verbatim}
(SETQ a 'hello)
(ATOM a)
\end{verbatim}

\item
\begin{verbatim}
(SETQ b '((what) is a list "?"))
(CONS (CAR b) (CDR b))
(SETQ (CAR b) (CAR (CAR b)))
\end{verbatim}

\item
\begin{verbatim}
(SET 'c '(21 22 23 50))
(SETQ (CDR (CDR c)) (CONS 33 NIL))
(REVERSE c)
\end{verbatim}

\end{enumerate}

\pagebreak

\noindent \textbf{Answers}

\begin{enumerate}

\item
\begin{verbatim}
(DIV 24 -3)
-8
\end{verbatim}

\item ~ %% \vspace{-20pt}
  \\ \begin{tabular}{p{5in}}
\begin{verbatim}
(ADD (MULT 48 40) (SUB 81 1) (MULT 3 5))
(ADD 1920 80 15)
2015
\end{verbatim}
\end{tabular}

\item ~ %% \vspace{-10pt}
  \\ \begin{tabular}{l l}
\texttt{(SETQ a 'hello)} & Set variable \texttt{a} to symbol \texttt{hello}. \\
\texttt{(ATOM a)}        & Returns true, since \texttt{a} is not a list.
\end{tabular}

\item \begin{alignat*}{5}
  & 2 & 1 & \text{A} & 7 \\
  - & & 1 & 1 & 0 \\[-1.2em]
  \cline{2-5} \\[-2em]
  & 2 & 0 & 9 & 7
\end{alignat*}

\item \begin{alignat*}{6}
  & & \text{\scriptsize 1} & & & \\[-0.5em]
  & 1 & 0 & 5 & 3 & 1 \\
  + & 1 & 2 & 4 & 1 & 4 \\[-1.2em]
  \cline{2-6} \\[-2em]
  & 2 & 3 & 1 & 4 & 5
\end{alignat*}

\item \begin{alignat*}{8}
  & & \text{\tiny \textsuperscript{1}0} & & & & & \\[-0.5em]
  - & & 1 & \hspace{2pt} 1 & 1 & 0 & 1 & 1 \\[-1.2em]
  \cline{2-8} \\[-2em]
  & & 1 & \hspace{2pt} 1 & 0 & 1 & 0 & 0
\end{alignat*}

\item \begin{align*}
  3\text{C}0_{16} &= 0011\ 1100\ 0000_2
\end{align*}

  \begin{align*}
    340_8 &= 011\ 100\ 000_2
  \end{align*}

  \begin{alignat*}{12}
    & \text{\scriptsize 1} & \text{\scriptsize 1} & \text{\scriptsize 1} & \text{\scriptsize 1} & & & & & & & \\
    & & 1 & 1 & 1 & 1 & 0 & 0 & 0 & 0 & 0 & 0 \\
    + & & & & 1 & 1 & 1 & 0 & 0 & 0 & 0 & 0 \\[-1.2em]
    \cline{2-12} \\[-2em]
    & 1 & 0 & 0 & 1 & 0 & 1 & 0 & 0 & 0 & 0 & 0
  \end{alignat*}

  \begin{align*}
    10010100000_2 &= 100\ 1010\ 0000_2 \\
    &= 4\text{A}0_{16}
  \end{align*}

\item \begin{align*}
  6\text{FAC}_{16} &= 0110\ 1111\ 1010\ 1100_2 \\
  &= 01101\ 111\ 101\ 01100
\end{align*}

  so

  \begin{align*}
    \text{Field A} &= \text{D}_{16} \\
    \textbf{Field B} &= 7_{16} \\
    \text{Field C} &= 5_{16} \\
    \text{Field D} &= \text{C}_{16}
  \end{align*}

\end{enumerate}


\end{document}
