\documentclass[12pt,letterpaper]{article}

\usepackage{amsmath}

\begin{document}

\begin{center}
  LISP Programming Worksheet

  Created by Christopher Cooper and Sam Craig for the West Lafayette High School ACSL Club of 2014--2015
\end{center}

\textbf{You may not need to know all of this material for the next contest!}
Problems 1 and 2 are representative of problems you may see on contest \#2.
Problems 3 through 5 are representative of problems you may see on contest \#4.

\bigskip
\noindent \textbf{Questions}

Evaluate the following LISP code.

\begin{enumerate}

\item
\begin{verbatim}
(DIV 24 (SUB 2 5))
\end{verbatim}

\item
\begin{verbatim}
(ADD (MULT 48 40)
     (SUB (SQUARE 9) 1)
     (MULT 3 5))
\end{verbatim}

\item
\begin{verbatim}
(SETQ a 'hello)
(ATOM a)
\end{verbatim}

\item
\begin{verbatim}
(SETQ b '((what) is a list "?"))
(CONS (CAR b) (CDR b))
(SETQ (CAR b) (CAR (CAR b)))
\end{verbatim}

\item
\begin{verbatim}
(SET 'c '(21 22 23 50))
(SETQ (CDR (CDR c)) (CONS 33 NIL))
(REVERSE c)
\end{verbatim}

\end{enumerate}

\pagebreak

\noindent \textbf{Answers}

\begin{enumerate}

\item
\begin{verbatim}
(DIV 24 -3)
-8
\end{verbatim}

\item ~ %% \vspace{-20pt}
  \\ \begin{tabular}{p{5in}}
\begin{verbatim}
(ADD (MULT 48 40) (SUB 81 1) (MULT 3 5))
(ADD 1920 80 15)
2015
\end{verbatim}
\end{tabular}

\item ~ %% \vspace{-10pt}
  \\ \begin{tabular}{l l}
\texttt{(SETQ a 'hello)} & Set variable \texttt{a} to symbol \texttt{hello}. \\
\texttt{(ATOM a)}        & Returns true, since \texttt{a} is not a list.
  \end{tabular}
  
\end{enumerate}

\end{document}
