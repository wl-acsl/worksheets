\documentclass[12pt,letterpaper,fleqn]{article}

\usepackage{amsmath}
\usepackage{cancel}

\begin{document}

\begin{center}
  Computer Number Systems Worksheet

  Created by Sam Craig for the West Lafayette High School ACSL Club of 2014--2015
\end{center}

A subscript after a number designates what base it is in.
For example, $20_{10}$ is 20 in base ten, but usually base ten is assumed and left out.
$10_{16}$ is 10 in base 16, or 16 in base ten.
A \textbf{bit} is a \textbf{bi}nary (base 2) digi\textbf{t}.

\bigskip
\noindent \textbf{Questions}

\begin{enumerate}

\item Convert $15\text{F}_{16}$ to octal.

\item Convert $101000100011_2$ to hexadecimal.

\item Evaluate $21\text{A}7_{16} - 110_{16}$ in hexadecimal.

\item Evaluate $10531_8 + 12414_8$ in octal.

\item Evaluate $1101111_2 - 111011_2$ in binary.

\item Evaluate $3\text{C}0_{16} + 340_8$ in hexadecimal.

  %% This should be replaced with a non-packet problem in the future.
\item In the Technology Department's new computer, each ``word'' of memory
  contains 16 bits representing 4 pieces of information.
  The first 5 bits represent Field A; the next
  3 bits, Field B; the next 3 bits, Field C;
  and the last 5 bits, Field D.
  For example, the 16 bits comprising the ``word''
  $9699_{16}$ has fields with values of $19_{16}$, $4_{16}$,
  $6_{16}$, and $\text{A}_{16}$. What is Field C in $6\text{FAC}_{16}$? (Express your answer as
  a base 16 number.)

\end{enumerate}

\pagebreak

\noindent \textbf{Answers}

\begin{enumerate}

\item \begin{align*}
  15\text{F}_{16} &= 0001\ 0101\ 1111_2 \\
  &= 101\ 011\ 111_2 \\
  &= 537_8
\end{align*}
  
\item \begin{align*}
  101000100011_2 &= 1010\ 0010\ 0011_2 \\
  &= \text{A}23_{16}
\end{align*}

\item \begin{alignat*}{5}
  & 2 & 1 & \text{A} & 7 \\
  - & & 1 & 1 & 0 \\[-1.2em]
  \cline{2-5} \\[-2em]
  & 2 & 0 & 9 & 7
\end{alignat*}

\item \begin{alignat*}{6}
  & & \text{\scriptsize 1} & & & \\[-0.5em]
  & 1 & 0 & 5 & 3 & 1 \\
  + & 1 & 2 & 4 & 1 & 4 \\[-1.2em]
  \cline{2-6} \\[-2em]
  & 2 & 3 & 1 & 4 & 5
\end{alignat*}

\item \begin{alignat*}{8}
  & & \text{\tiny \textsuperscript{1}0} & & & & & \\[-0.5em]
  & \cancel{1} & \cancel{1} & ^{\text{\tiny 1}}0 & 1 & 1 & 1 & 1 \\
  - & & 1 & \hspace{2pt} 1 & 1 & 0 & 1 & 1 \\[-1.2em]
  \cline{2-8} \\[-2em]
  & & 1 & \hspace{2pt} 1 & 0 & 1 & 0 & 0
\end{alignat*}

\item \begin{align*}
  3\text{C}0_{16} &= 0011\ 1100\ 0000_2
\end{align*}

  \begin{align*}
    340_8 &= 011\ 100\ 000_2
  \end{align*}

  \begin{alignat*}{12}
    & \text{\scriptsize 1} & \text{\scriptsize 1} & \text{\scriptsize 1} & \text{\scriptsize 1} & & & & & & & \\
    & & 1 & 1 & 1 & 1 & 0 & 0 & 0 & 0 & 0 & 0 \\
    + & & & & 1 & 1 & 1 & 0 & 0 & 0 & 0 & 0 \\[-1.2em]
    \cline{2-11} \\[-2em]
    & 1 & 0 & 0 & 1 & 0 & 1 & 0 & 0 & 0 & 0 & 0
  \end{alignat*}

\item \begin{align*}
  6\text{FAC}_{16} &= 0110\ 1111\ 1010\ 1100_2 \\
  &= 01101\ 111\ 101\ 01100
\end{align*}

  so

  \begin{align*}
    \text{Field A} &= \text{D}_{16} \\
    \textbf{Field B} &= 7_{16} \\
    \text{Field C} &= 5_{16} \\
    \text{Field D} &= \text{C}_{16}
  \end{align*}

\end{enumerate}

\end{document}
